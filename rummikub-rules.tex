\documentclass{article}
\usepackage{caption}
\usepackage{geometry}
\geometry{letterpaper, landscape, margin=0.5in}
\usepackage{multicol}

\begin{document}
\title{\huge Rummikub\\\medskip
\normalsize An Explanation of Family Rules\\
\bigskip
\bigskip
\small First Edition
}
\date{\small\today}

\begin{multicols*}{3}
\maketitle
\section*{Objective}
In each \textsc{game} of Rummikub, the player's objective is to \textsc{go out}
	by playing every tile in their rack. At the end of the \textsc{session},
	the \textsc{score} is calculated and one player wins.

\section*{Setup}

Rummikub requires ample space for both the tiles that have been played (the
	\textsc{board}) and for the face-down tiles which have not yet been
	drawn (the \textsc{pile}). Make sure that players can see the entire
	\textsc{board} and can easily reach the \textsc{pile}.

\subsection*{Scoring}

Before starting a \textsc{session}, players should agree on a method for keeping
	\textsc{score}. They may also elect a scorekeeper for their convenience.
	Popular scorekeeping methods include:
\begin{itemize}
	\item \textit{Points} --- At the end of each \textsc{game}, players
		total up the value of their remaining tiles and add it to a
		running total for the \textsc{session}. When the
		\textsc{session} ends, the player with the \textit{least} points
		wins.
	\item \textit{Games} --- For every \textsc{game} in which a player
		\textsc{goes out}, that player receives one point towards a
		running total. When the \textsc{session} ends, the player with
		the \textit{most} points wins.
\end{itemize}

\noindent After deciding how to keep \textsc{score}, the players may optionally
	\footnote{Often, the end of the \textsc{session} is decided while
	underway and/or in an ad-hoc manner (\textit{e.g.} people are tired of
	playing, a player is leaving, \textit{etc.})} decide how many
	\textsc{games} will be played in the \textsc{session}.

\medskip
\noindent Next, take the following steps:

\begin{enumerate}
	\item Place the tiles face-down on the table and mix them thoroughly.
	\item \label{itm:setup-first-player}Each player picks a tile (highest
		number wins) to determine who will go first.
	\item Return the tiles to the table and mix thoroughly.
	\item Each player takes 14 tiles and places them on their rack.
	\item Clear a space in the center of the table for the \textsc{board}.
		Group the remaining face-down tiles of the \textsc{pile}
		wherever convenient.
\end{enumerate}

\section*{Gameplay}
Starting with the player who won the draw, each player takes their turn before
	play passes clockwise. Each turn, a player must either \textsc{play} or
	\textsc{draw} a tile. Players who cannot make a legal \textsc{play} must
	\textsc{draw}.\footnote{Technically, players who can make a legal
	\textsc{play} can still opt to \textsc{draw}.}

\subsection*{Tile Arrangements}

At the end of every player's turn, the \textsc{board} must be in a legal state.
A legal \textsc{board} is comprised of one or more of the following
\textsc{sets}:

\begin{itemize}
	\item \textsc{group} --- A set of three (3) or more tiles of the same
		value and different colors. Colors cannot be repeated, so the
		largest possible \textsc{group} is four tiles (one of each
		color).
	\item \textsc{run} --- A set of three (3) or more tiles with sequential
		values and the same color. \textsc{runs} can start and end on
		any number. The largest possible sequence contains one of each
		number (1 through 13).
\end{itemize}

\subsection*{Making a Play}

A \textsc{play} is a sequence of one or more actions to change the state of the
\textsc{board}.

\begin{itemize}
	\item \textit{\textsc{melding}} --- Each \textsc{game}, each player must
		meld before taking any other actions. Once a player has
		\textsc{melded}, their \textsc{play} for that turn can include
		other actions in this list. See Melding for more detail.
	\item \textit{Placing a \textsc{set}} --- Place three or more tiles onto
		the \textsc{board} to form a legal \textsc{set}.
	\item \textit{Adding to a \textsc{set}} --- Add any number of non-joker
		tiles to an existing \textsc{set} to form a new legal
		\textsc{set}. See The Joker for more detail.
	\item \textit{Rearranging \textsc{sets}} --- Rearrange one or more
		existing \textsc{sets} into a new legal configuration. Existing
		\textsc{sets} containing jokers have restrictions. See The Joker
		for more detail.
	\item \textit{Reusing a joker} --- See The Joker - Reuse for details on
		how to replace and reuse a joker.
\end{itemize}

Players are generally allowed to revise their actions any number of times before
completing their \textsc{play}. As long as the special constraints around
\textsc{melding} and the use of the joker are honored, the main requirement is
that the \textsc{board} is left in a legal state when the \textsc{play} ends.

Once a player has finished their \textsc{play}, they should verbally indicate to
the next player that they have completed their turn.

\subsection*{Melding}
\label{sec:melding}

At the beginning of each \textsc{game}, players must first \textsc{meld} before
they can participate in regular play. If a player does not have the necessary
tiles to \textsc{meld}, they must \textsc{draw}.

To \textsc{meld}, players must form one or more legal \textsc{sets} \textit{only
using tiles from their rack}. The sum of the face value of the tiles used must
meet or exceed 30 points\footnote{If a \textsc{set} used to \textsc{meld}
contains a joker, its face value is counted towards the sum as if it was the
tile it represents.}. Place the sets on the \textsc{board} to complete the
\textsc{meld}.

As indicated in Making a Play, once a player has \textsc{melded}, they may
continue to take other actions as a part of their \textsc{play}.

\subsection*{The Joker}
\label{sec:joker}

The two joker tiles are used to substitute as a natural tile of any value and
color. Other than the restrictions outlined in this section, jokers function as
if they were the tiles they represent.

\subsubsection*{Placement}

Unlike natural tiles, jokers cannot be added to an existing set from a player's
rack. The only way to play a joker from the rack is to add it to the
\textsc{board} as part of a new \textsc{set} of 3 or more tiles\footnote{In the
unlikely event that a player has both jokers on their rack, they must satisfy
the requirements for each joker separately. For example, two natural tiles
cannot satisfy the requirement for both jokers as part of a 4-tile
\textsc{set}.}.

\subsubsection*{Reuse}

Once a joker is on the \textsc{board}, any player can replace a joker with the
natural tile it replaces\footnote{In a 3-tile \textsc{group}, a joker will have
an ambiguous color representation. A player may replace it with a natural tile
of either color.} and reuse the joker in another \textsc{set}.

To accomplish this, a player must do both of the following:

\begin{itemize}
	\item Provide a tile of the correct number and color from their rack.
		This tile will replace the joker in its current position.
	\item Provide at least two additional tiles of the correct number and
		color from their rack. These tiles will form a new \textsc{set}
		along with the joker.
\end{itemize}

In the event that more than one joker is present in a \textsc{set}, each joker
must be substituted individually as if it was the only one.

\subsubsection*{Rearrangement}

Unlike \textsc{sets} with only natural tiles, \textsc{sets} containing a joker
must meet two additional restrictions:

\begin{itemize}
	\item Rearranging the \textsc{set} must be done one tile at a time and
		must remain a legal \textsc{set} each step of the
		way.\footnote{The rest of the \textsc{board} may be temporarily
		illegal during the rearrangement, but not any \textsc{sets}
		containing jokers.}
	\item If the \textsc{set} starts the rearrangement as a \textsc{run}, it
		cannot end it as a \textsc{group} (or vice versa).
\end{itemize}

The purpose of these restrictions is to ensure that the only way to reuse a
joker is to meet the requirements outlined above.

\medskip
\noindent\textit{Example 1.} Imagine the following \textsc{run} of tiles
containing a joker: 5 (orange), 6 (orange), joker. Here are some examples for
how to modify it:

\begin{itemize}
	\item \textit{Legal} --- Adding 8 (orange) and 9 (orange) to the
		\textsc{run} and removing 5 (orange) and 6 (orange) to use
		elsewhere. Tile by tile, the \textsc{run} remains legal and the
		joker continues to represent a 7 (orange) in a \textsc{run}.
	\item \textit{Legal} --- Merging an existing \textsc{run} containing 2
		(orange), 3 (orange) and 4 (orange). The resulting run is now 6
		tiles.\footnote{It is technically legal to merge two
		\textsc{runs}, each containing a joker. Combined with
		rearrangements, some \textsc{runs} that are illegal to place may
		legally exist on the board.}
	\item \textit{Illegal} --- Adding 7 (blue) and 7 (black) and removing 5
		(orange) and 6 (orange). Even though the joker ends up in a
		legal \textsc{set}, it participated in an illegal \textsc{set}
		as tiles were rearranged.
	\item \textit{Illegal} --- Merging an existing \textsc{run} containing a
		joker, 9 (orange), 10 (orange). Then, rearrange 9 (orange), 10
		(orange), and 5 (orange) elsewhere on the board. Finally, add 5
		(blue). This rearrangement is (perhaps surprisingly) legal up
		until the addition of the 5 (blue). Adding the 5 (blue) would
		cause the resulting \textsc{set} to become a \textsc{group} when
		it started as a \textsc{run}.
\end{itemize}

\noindent\textit{Example 2.} Imagine the following \textsc{group} of tiles
containing a joker: 11 (orange), 11 (red), joker. Here are some examples for how
to modify it:

\begin{itemize}
	\item \textit{Legal} --- Add 11 (blue) and remove 11 (orange). Even
		though the joker has an ambiguous color at the beginning (blue,
		black) and a different ambiguity at the end (orange, black), the
		set remained legal as tiles were added and subtracted.
	\item \textit{Illegal} --- Add 10 (orange) and 10 (red) and remove 11
		(orange) and 11 (red). Even though the joker ends in a legal
		\textsc{set}, it participated in an illegal \textsc{set} as
		tiles were rearranged.
\end{itemize}

\subsubsection*{Scoring}

If a player still has a joker on their rack at the end of a \textsc{game}, it is
treated as a tile with a value of 30 points for the purposes of scoring.

\section*{Ending a Game}

When a player \textsc{plays} the last tile on their rack, they have \textsc{gone
out} and the game is over. According to the chosen methodology (see Scoring),
calculate \textsc{score}.

If necessary, reset the \textsc{board} for the next game (see Setup). The player
who has \textsc{gone out} has the privilege of going first in the next game. If
there is no next game, the \textsc{session} ends and a winner is determined.

\end{multicols*}
\end{document}
